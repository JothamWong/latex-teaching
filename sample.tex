\documentclass[12pt]{article}

\title{Sample \LaTeX Document}
\author{Jotham Wong Yi Shuen}
\date{3 September 2024}

\usepackage{amsmath}
\usepackage{setspace}
\usepackage{minted}
\usepackage{mathtools}

\begin{document}
    \maketitle
    \tableofcontents

    \section{Spaces}
    It does not matter whether you
    enter one or several             spaces
    after a word.

    An empty line starts a new
    paragraph.
    
    Forcing a linebreak\\
    Ooga BOOGA

    The diacritic \^j vs the \^{}

    \section{Section}
    \subsection{Subsection}
    \subsubsection{Subsubsection}
    Cannot go any lower.
    \subsubsection*{unnumbered subsubsection}
    I have no number!

    \section{Font Styles}

    We can change fonts using the following commands, \emph{Emphasized}, \texttt{Monospaced Font}, \textsc{Small Capitals}, \uppercase{uppercase}

    \section{Font Size}

    Within the same scope, we can change the font size as follows {\tiny me is tiny}, {\scriptsize me is ooga}, {\small smally}, {\normalsize normal}, {\LARGE OOGA BOOGA}, {\Huge OOGA BUNGA}

    \section{Non-breaking space}
    How we can use the \~{} to do a non-breaking space. Observe that this reallylongword will break. Now let's use \~{}.

    How we can use the \~{} to do a non-breaking space. Observe that this really~longword will break. Now let's use \~{}.

    But in practice, I would not recommend this. Just let \LaTeX~do the formatting for you.
    Unless it really breaks your immersion.
    We can also use this in math equations.

    Compare the two lines below
    
    
    $f(x) = x^2 \quad \text{for} x > 0$

    $f(x) = x^2 \quad \text{for}~x > 0$

    \section{Line Spacing}

        ~ Jotham Wong   

        Jotham\singlespacing Wong
        Jotham\onehalfspace Wong

        \begin{spacing}{2.5}
            This paragraph has \\ huge gaps \\ between lines.
        \end{spacing}

    \section{Quote-marks}
    `quote' vs ``quote''

    \section{Paragraph Alignment}

        \begin{flushleft}
            I am flushed left.    
        \end{flushleft}


        \begin{flushright}
            I am flushed right.    
        \end{flushright}

        \begin{center}
            Hello I am centered
        \end{center}


    \section{Verbatim}
        
        We begin the verbatim environment.

        \begin{verbatim}
            The verbatim environment
                simply reproduces every
            character you input,
            including all s p a c e s!
        \end{verbatim}

    \section{Code Blocks}

        Using \texttt{minted} package to write code blocks.

        \begin{minted}{python}
            # This is python code
            def factorial(n):
                return n == 1 ? 1 : n * factorial(n-1)
        \end{minted}

        \begin{minted}{java}
            // This is java code
            public int factorial(int n) {
                if (n == 1) {
                    return 1;
                } else {
                    return n * factorial(n-1);
                }
            }
        \end{minted}

        Notice the syntax highlighting. (Basically verbatim + syntax)

    \section{Math fun}
        Short hand equations: \[e^{i \pi} + 1 = 5\]

        The equivalent with explicit environment command (and numbered equation)

        \begin{equation}
            e^{i \pi} + 1 = 0
        \end{equation}
    
    \section{Citing stuff}
        \LaTeX is a set of macros built upon \TeX \cite{texbook}.
    
    \bibliographystyle{plain}
    \bibliography{ref}
\end{document}